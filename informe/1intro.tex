\section{Objetivo del trabajo}

El objetivo de este trabajo es utilizar técnicas provistas por la teoría de la información para distinguir
diversos aspectos de la red de manera analítica. Para ello, es sugerido el uso de dos herramientas modernas
de manipulación y análisis de paquetes: Wireshark y Scapy.

\subsection{Breve introducción}

En primer lugar implementamos herramientas para simular fuentes de informaci\'on. Una fuente de informaci\'on, desde el punto de vista de la teor\'ia de la informaci\'on de Shannon, es un objeto que emite s\'imbolos $s_i$, en donde cada uno tiene una probabilidad $p_i$ de ser emitido. Entre ellas tendremos:

\begin{itemize}
\item[$\circ$]Una herramienta que simula una fuente que emite paquetes Ethernet, es decir cada s\'imbolo en este caso ser\'a un posible frame generado con dicho protocolo.
\item[$\circ$]Otra herramienta que sumila una fuente que emite paquetes que utilizan el protocolo ARP.
\end{itemize}


