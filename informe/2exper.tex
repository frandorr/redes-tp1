\section{El problema}
En este trabajo práctico se utilizarán dos modelos de redes neuronales artificiales 
(ambos basados en variantes del aprendizaje hebbiano no supervisado). Ambos se utilizarán 
sobre el mismo conjunto de datos, y ambos con el objetivo de clasificar los datos. 

\subsection{Los datos del problema}
Los datos consisten en descripciones de compañías Brasileñas, estas clasificadas en 9 categorías, 
según la actividad principal de la compañía. Cada una de las descripciones ha sido preprocesada, 
obteniendo una representación del estilo Bag-Of-Words. De modo que cada descripción ha sido 
transformada en un vector, donde el valor de una descripción en la $i$-ésima coordenada indica la cantidad de 
veces que aparece la $i$-ésima palabra en dicha descripción. Las palabras más comunes como conectores, 
artículos y preposiciones se han omitido en este preprocesamiento, por el poco aporte que hacen 
a la descripción en sí. 

De este modo, disponemos de 900 entradas, cada una con más de 856 atributos (coordenadas, es decir, 
las palabras que fueron tomadas en cuenta en el preprocesamiento) distribuidas uniformemente en las 9 categorías. 
Cada entrada está etiquetada con la categoría a la que pertenece. 

A continuación presentaremos el trabajo realizado con los dos modelos de 
redes neuronales artificiales que se usaron. El primero, un modelo de análisis de 
componentes principales (reducción de dimensiones) usando las reglas de Oja y Sanger. El segundo, 
un modelo de mapeo de características auto-organizado. 
